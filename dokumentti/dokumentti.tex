\documentclass[12pt] {article}
\usepackage[utf8] {inputenc}
\usepackage[finnish] {babel}
\usepackage {amsmath}
\usepackage {amssymb}
\usepackage {amsthm}
\usepackage {mathtools}
\usepackage {graphicx}

\begin {document}

\title {Projektin dokumentti}
\author {Miska Kananen (652102, TiK, 1.vk)}
\date {\today}
\maketitle

\tableofcontents

\section {Yleiskuvaus}

Toteutin ohjelman, jolla voi renderöidä kolmioista koostuvia kolmiulotteisia malleja reaaliajassa. Mallit voivat olla mielivaltaisia, eivätkä ne ole rajoittuneita pelkästään suorakulmaisiin seiniin. Mallin ympärillä voi liikkua vapaasti ja kameraa voi kääntää mielivaltaiseen suuntaan.

\section {Käyttöohje}



\section {Ohjelman rakenne}

\section {Algoritmit}

\section {Tietorakenteet}

\section {Tiedostot}

\section {Testaus}

\section {Puutteet ja viat}

\section {Parhaat ja heikoimmat kohdat}

\section {Poikkeamat suunnitelmasta}

\section {Kokonaisarvio}

\section {Viitteet}

\end {document}
