\documentclass[12pt] {article}
\usepackage[utf8] {inputenc}
\usepackage[finnish] {babel}
\usepackage {amsmath}
\usepackage {amssymb}
\usepackage {amsthm}
\usepackage {mathtools}
\usepackage {graphicx}

\begin {document}

\title {Projektin dokumentti}
\author {Miska Kananen (652102, TiK, 1.vk)}
\date {\today}
\maketitle

\tableofcontents

\section {Yleiskuvaus}

Toteutin ohjelman, jolla voi renderöidä reaaliajassa kolmioista koostuvia maailmaan sijoitettuja kolmiulotteisia malleja. Mallit voivat olla mielivaltaisia, eivätkä ne ole rajoittuneita esimerkiksi pelkästään suorakulmaisiin seiniin. Mallin ympärillä voi liikkua vapaasti ja kameraa voi kääntää mielivaltaiseen suuntaan.

\section {Käyttöohje}

Kun ohjelma käynnistetään, se lataa mallinnettavan maailman \texttt{world}-nimisestä tekstitiedostosta ohjelman juurikansiosta. Jos lataaminen epäonnistuu, tulostuu tieto tästä standarditulosteeseen.

Kameraa voi liikuttaa eteen, taakse ja sivuille \texttt{W, A, S, D}-näppäimillä. \texttt{Q} ja \texttt{Z} liikuttavat kameraa ylös ja alas. Nuolinäppäimet kääntävät kameraa.

\section {Ohjelman rakenne}

\section {Algoritmit}

\section {Tietorakenteet}

\section {Tiedostot}

\section {Testaus}

\section {Puutteet ja viat}

\section {Parhaat ja heikoimmat kohdat}

\section {Poikkeamat suunnitelmasta}

\section {Kokonaisarvio}

\section {Viitteet}

\end {document}
